\chapter{Les Tests}

Pour rappel, tous les tests sont présents dans un seul fichier : tests.c .
Tout d'abord, je décris comment faire fonctionner le programme, puis je donne une liste détaillée des fonctions présentes dans mon programme.

\section{Comment faire marcher le programme}

L'ensemble des fonctions est testé via un seul fichier exécutable. Il n'y a qu'à faire make ; ./tests .
Il n'y a qu'à lire ce qui est affiché à l'écran.
Un menu a été mis en place pour naviguer entre les différents tests possibles.

\section{Concernant les sources du programme de tests}
\subsection{Les procédures de confort}

Voici donc une liste des fonctions que j'utilise à des fins de confort : 
\begin{itemize}
	\item choix-nbl-nbc
	\item creation-matrice
	\item print-menu
	\item menu
\end{itemize}

Ces fonctions sont simples et me permettent soit de factoriser mon code soit d'unifier l'affichage.

\subsection{Les procédures de test à proprement parler}

\begin{itemize}
	\item test-determinant-comatrice-transposee
	\item test-multiplication
	\item test-addition-matrices
	\item test-inversion
	\item test-addition-lignes
	\item tests-divers
	\item test-systeme-gauss
\end{itemize}

Chaque test demande de créer une ou deux matrices. Le reste est affiché.

\clearpage
