\chapter{Conclusion}

\section{Ce qui fonctionne}

Les procédures basiques que j'ai testé maintes fois : 
\begin{itemize}
	\item l'addition ;
	\item la multiplication ;
	\item la création d'une comatrice ;
	\item la transposée.
\end{itemize}

Les procédures plus complexes qui ont été testé avec des exemples vus en cours :
\begin{itemize}
	\item inversion via pivot de Gauss ;
	\item inversion via comatrice ;
\end{itemize}

À savoir également que j'ai fait énormément de fonctions qui sont des outils me permettant de répondre aux questions plus complexes.
Cependant, je n'ai pas eu le temps de tout implanter, mais les outils sont là. 
Comme par exemple les "pmatrice" qui sont des matrices de polynômes avec plusieurs fonctions qui y sont liées comme l'addition, la soustraction et la multiplication de polynômes.

Pour la fonction de pivot de Gauss, si une ligne est égale à une autre ligne (ou une multiplication de celle-ci) alors ça ne fonctionne pas.
J'ai fait en sorte que lorsque la matrice de gauche ne devient pas une matrice identitée alors ma fonction renvoie -1 (synonyme d'une erreur).

Ce programme est "garanti sans fuites de mémoire" ©.

\section{Ce qui fonctionne moins}

Concernant les valeurs propres, il faut que l'on rentre une matrice A qui ait déjà une ligne (ou une colonne) avec deux valeurs = 0 (et non sur la diagonale).

La seconde partie de la question (résolution par pivot des valeurs propres) a été implanté dans la précipitation. Je n'ai pas fait énormément de tests.
Ceci est juste par manque de temps.

\section{Ce qui n'a pas été implémenté}
\begin{itemize}
	\item Décomposition QR.
\end{itemize}
