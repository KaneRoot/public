\chapter{Les fichiers du projet}

Les noms des fichiers sont assez explicites pour ne pas avoir à expliquer plus en détail leur contenu. 
Cependant si vous souhaitez en savoir plus, les fichiers ".h" sont suffisamment commentés pour servir de manuel.

\section{Les sources du programme}

Voici une liste simple des fichiers du projet.


\begin{itemize}
	\item base.c
	\item base.h
\end{itemize}
Permet la création des différentes structures, les polynômes, les matrices et les matrices de polynômes.
Permet également la destruction propre de ces mêmes structures et leur affichage.

\begin{itemize}
	\item calculs base.c
	\item calculs base.h
\end{itemize}
Permet de faire des additions, multiplications de lignes sur une matrice (éventuellement de polynômes) ainsi que des additions et multiplications de polynômes.

\begin{itemize}
	\item determinant.c
	\item determinant.h
\end{itemize}
Calcul du déterminant d'une matrice (det nxn).

\begin{itemize}
	\item comatrice.c
	\item comatrice.h
\end{itemize}
Calcul de la comatrice (utilise le déterminant).

\begin{itemize}
	\item pivot-gauss.c
	\item pivot-gauss.h
\end{itemize}
Effectue le pivot de Gauss sur une matrice en répercutant chaque addition de lignes, inversion de lignes etc. sur une seconde matrice passée en parallèle.
Une fonction présente également est la recherche du pivot suivant dans une matrice à un emplacement donné.

\begin{itemize}
	\item inversion.c
	\item inversion.h
\end{itemize}
Calcule l'inversion d'une matrice par pivot de Gauss et par comatrice.

\begin{itemize}
	\item valeurs-propres.c
	\item valeurs-propres.h
\end{itemize}
Calcul des valeurs propres.

\begin{itemize}
	\item tests.c
\end{itemize}
L'ensemble des tests possibles sont présents dans ce fichier.

\begin{itemize}
	\item Makefile
\end{itemize}
Produit un unique exécutable via la commande "make".

\section{Les sources du rapport}


\begin{itemize}
	\item biblio.bib
	\item Conclusion.tex
	\item Fichiers.tex
	\item Global.pdf
	\item Global.tex
	\item Intro.tex
	\item Tests.tex
\end{itemize}

\clearpage
