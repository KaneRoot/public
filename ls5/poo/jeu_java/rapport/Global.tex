\documentclass[12pt,a4paper,utf8x]{report}
\usepackage [frenchb]{babel}

% Pour pouvoir utiliser 
\usepackage{ucs}
\usepackage[utf8x]{inputenc}

\usepackage{url} % Pour avoir de belles url
\usepackage {geometry}

% Pour mettre du code source
\usepackage {listings}
% Pour pouvoir passer en paysage
\usepackage{lscape}

% Pour pouvoir faire plusieurs colonnes
\usepackage {multicol}
% POur crééer un index
\usepackage{makeidx}
\makeindex

% Pour les entetes de page
% \usepackage{fancyheadings}
%\pagestyle{fancy}
%\renewcommand{\sectionmark}[1]{\markboth{#1}{}} 
%\renewcommand{\subsectionmark}[1]{\markright{#1}} 

% Pour l'interligne de 1.5
\usepackage {setspace}
% Pour les marges de la page
\geometry{a4paper, top=2.5cm, bottom=3.5cm, left=1.5cm, right=1.5cm, marginparwidth=1.2cm}

\parskip=5pt %% distance entre § (paragraphe)
\sloppy %% respecter toujours la marge de droite 

% Pour les pénalités :
\interfootnotelinepenalty=150 %note de bas de page
\widowpenalty=150 %% veuves et orphelines
\clubpenalty=150 

%Pour la longueur de l'indentation des paragraphes
\setlength{\parindent}{15mm}



%%%% debut macro pour enlever le nom chapitre %%%%
\makeatletter
\def\@makechapterhead#1{%
  \vspace*{50\p@}%
  {\parindent \z@ \raggedright \normalfont
    \interlinepenalty\@M
    \ifnum \c@secnumdepth >\m@ne
        \Huge\bfseries \thechapter\quad
    \fi
    \Huge \bfseries #1\par\nobreak
    \vskip 40\p@
  }}

\def\@makeschapterhead#1{%
  \vspace*{50\p@}%
  {\parindent \z@ \raggedright
    \normalfont
    \interlinepenalty\@M
    \Huge \bfseries  #1\par\nobreak
    \vskip 40\p@
  }}
\makeatother
%%%% fin macro %%%%

%Couverture 

\title
{
	\normalsize{Rapport de projet\\
	Université de Strasbourg\\
	2012}\\
	\vspace{15mm}
	\Huge{Rapport de fin de projet}
}
\author{Pittoli Philippe\\
		Kilian Hett
	\vspace{45mm}
}

%\date{	
	%\normalsize{Université de Strasbourg (encore ?)\\
	%\vspace{5mm}	
	%Je pense que ce sera tout pour la page de présentation.
	%}
%}

\begin{document}

\maketitle

%Remerciements

Je tiens à remercier :
et on met la liste des personnes que l'on remercie. Toto, tutu, titi. et on met la liste des personnes que l'on remercie. Toto, tutu, titi.et on met la liste des personnes que l'on remercie. Toto, tutu, titi.et on met la liste des personnes que l'on remercie. Toto, tutu, titi.


Et on met la liste des personnes que l'on remercie. Toto, tutu, titi.et on met la liste des personnes que l'on remercie. Toto, tutu, titi.et on met la liste des personnes que l'on remercie. Toto, tutu, titi.et on met la liste des personnes que l'on remercie. Toto, tutu, titi.et on met la liste des personnes que l'on remercie. Toto, tutu, titi.

%\clearpage

\tableofcontents
\clearpage

% Pour avoir un interligne de 1,5
\begin{onehalfspace}

\chapter{Introduction}

Voici mon rapport de fin de projet.
Il a été écrit avec \LaTeX ~et les sources sont disponibles sur le dépôt.

Le projet a consisté au développement d'une application de vente aux enchères en réseau.
Trois programmes distincts ont été développés, un programme de client enchérisseur, 
un programme destiné au vendeur et un serveur qui gère la vente.
Le langage utilisé a été le C.

Les sources sont également disponibles sur mon dépôt GIT \cite{GIT} public
\protect\footnote{
Github : https://github.com/KaneRoot/public
}.


\chapter{Le titre du chapitre}

\section{Le titre de la section qui va bien}

\subsection{Titre de la sous section}

Ici du texte et du blabla, ce que l'on veut dire et écrire. A remplacer. Ici du texte et du blabla, ce que l'on veut dire et écrire. On peut faire une citation \cite{Motclef1}.
A remplacer. Ici du texte et du blabla, ce que l'on veut dire et écrire. A remplacer. Ici du texte et du blabla, ce que l'on veut dire et écrire. A remplacer. Ici du texte et du blabla, ce que l'on veut dire et écrire. A remplacer. Ici du texte et du blabla, ce que l'on veut dire et écrire. A remplacer.

Ici du texte et du blabla, ce que l'on veut dire et écrire. A remplacer. Ici du texte et du blabla, ce que l'on veut dire et écrire. A remplacer.
Ici du texte et du blabla, ce que l'on veut dire et écrire. A remplacer. Ici du texte et du blabla, ce que l'on veut dire et écrire. A remplacer. Ici du texte et du blabla, ce que l'on veut dire et écrire. A remplacer. Ici du texte et du blabla, ce que l'on veut dire et écrire. A remplacer.

%-- Note de bas de page sur les stades
\protect\footnote{Par exemple, on peut faire un pied de page :
\begin{itemize}
\item avec une liste à puces ;
\item avec une liste à puces ;
\item avec une liste à puces.
\end{itemize}
}
%-- Fin Note de bas de page sur les stades

Ici du texte et du blabla, ce que l'on veut dire et écrire. A remplacer. Ici du texte et du blabla, ce que l'on veut dire et écrire. A remplacer. Ici du texte et du blabla, ce que l'on veut dire et écrire. A remplacer. Ici du texte et du blabla, ce que l'on veut dire et écrire. A remplacer. Ici du texte et du blabla, ce que l'on veut dire et écrire. A remplacer. Ici du texte et du blabla, ce que l'on veut dire et écrire. A remplacer.

\begin{itemize}
\item avec une liste à puces ;
\item avec une liste à puces ;
\item avec une liste à puces.
\end{itemize}

Ici du texte et du blabla, ce que l'on veut dire et écrire. A remplacer. Ici du texte et du blabla, ce que l'on veut dire et écrire. A remplacer. Ici du texte et du blabla, ce que l'on veut dire et écrire. A remplacer. Ici du texte et du blabla, ce que l'on veut dire et écrire. A remplacer. Ici du texte et du blabla, ce que l'on veut dire et écrire. A remplacer. Ici du texte et du blabla, ce que l'on veut dire et écrire. A remplacer.

\subsubsection{Titre de la sous sous section}

Ici du texte et du blabla, ce que l'on veut dire et écrire. A remplacer. Ici du texte et du blabla, ce que l'on veut dire et écrire. A remplacer. Ici du texte et du blabla, ce que l'on veut dire et écrire. A remplacer. Ici du texte et du blabla, ce que l'on veut dire et écrire. A remplacer. Ici du texte et du blabla, ce que l'on veut dire et écrire. A remplacer. Ici du texte et du blabla, ce que l'on veut dire et écrire. A remplacer.

Ici du texte et du blabla, ce que l'on veut dire et écrire. A remplacer. Ici du texte et du blabla, ce que l'on veut dire et écrire. A remplacer. Ici du texte et du blabla, ce que l'on veut dire et écrire. A remplacer. Ici du texte et du blabla, ce que l'on veut dire et écrire. A remplacer. Ici du texte et du blabla, ce que l'on veut dire et écrire. A remplacer. Ici du texte et du blabla, ce que l'on veut dire et écrire. A remplacer.

\subsubsection{Titre de la sous sous section}

Ici du texte et du blabla, ce que l'on veut dire et écrire. A remplacer. Ici du texte et du blabla, ce que l'on veut dire et écrire. A remplacer. Ici du texte et du blabla, ce que l'on veut dire et écrire. A remplacer. Ici du texte et du blabla, ce que l'on veut dire et écrire. A remplacer. Ici du texte et du blabla, ce que l'on veut dire et écrire. A remplacer. Ici du texte et du blabla, ce que l'on veut dire et écrire. A remplacer.

Ici du texte et du blabla, ce que l'on veut dire et écrire. A remplacer. Ici du texte et du blabla, ce que l'on veut dire et écrire. A remplacer. Ici du texte et du blabla, ce que l'on veut dire et écrire. A remplacer. Ici du texte et du blabla, ce que l'on veut dire et écrire. A remplacer. Ici du texte et du blabla, ce que l'on veut dire et écrire. A remplacer. Ici du texte et du blabla, ce que l'on veut dire et écrire. A remplacer.

\subsection{Conclusion}

Ici du texte et du blabla, ce que l'on veut dire et écrire. A remplacer. Ici du texte et du blabla, ce que l'on veut dire et écrire. A remplacer. Ici du texte et du blabla, ce que l'on veut dire et écrire. A remplacer. Ici du texte et du blabla, ce que l'on veut dire et écrire. A remplacer. Ici du texte et du blabla, ce que l'on veut dire et écrire. A remplacer. Ici du texte et du blabla, ce que l'on veut dire et écrire. A remplacer.

Ici du texte et du blabla, ce que l'on veut dire et écrire. A remplacer. Ici du texte et du blabla, ce que l'on veut dire et écrire. A remplacer. Ici du texte et du blabla, ce que l'on veut dire et écrire. A remplacer. Ici du texte et du blabla, ce que l'on veut dire et écrire. A remplacer. Ici du texte et du blabla, ce que l'on veut dire et écrire. A remplacer. Ici du texte et du blabla, ce que l'on veut dire et écrire. A remplacer.

\subsection{Titre de la sous section}

Ici du texte et du blabla, ce que l'on veut dire et écrire. A remplacer. Ici du texte et du blabla, ce que l'on veut dire et écrire. On peut faire une citation \cite{Motclef1}.
A remplacer. Ici du texte et du blabla, ce que l'on veut dire et écrire. A remplacer. Ici du texte et du blabla, ce que l'on veut dire et écrire. A remplacer. Ici du texte et du blabla, ce que l'on veut dire et écrire. A remplacer. Ici du texte et du blabla, ce que l'on veut dire et écrire. A remplacer.

Ici du texte et du blabla, ce que l'on veut dire et écrire. A remplacer. Ici du texte et du blabla, ce que l'on veut dire et écrire. A remplacer.
Ici du texte et du blabla, ce que l'on veut dire et écrire. A remplacer. Ici du texte et du blabla, ce que l'on veut dire et écrire. A remplacer. Ici du texte et du blabla, ce que l'on veut dire et écrire. A remplacer. Ici du texte et du blabla, ce que l'on veut dire et écrire. A remplacer.

\subsection{Titre de la sous section}

Ici du texte et du blabla, ce que l'on veut dire et écrire. A remplacer. Ici du texte et du blabla, ce que l'on veut dire et écrire. On peut faire une citation \cite{Motclef1}.
A remplacer. Ici du texte et du blabla, ce que l'on veut dire et écrire. A remplacer. Ici du texte et du blabla, ce que l'on veut dire et écrire. A remplacer. Ici du texte et du blabla, ce que l'on veut dire et écrire. A remplacer. Ici du texte et du blabla, ce que l'on veut dire et écrire. A remplacer.

Ici du texte et du blabla, ce que l'on veut dire et écrire. A remplacer. Ici du texte et du blabla, ce que l'on veut dire et écrire. A remplacer.
Ici du texte et du blabla, ce que l'on veut dire et écrire. A remplacer. Ici du texte et du blabla, ce que l'on veut dire et écrire. A remplacer. Ici du texte et du blabla, ce que l'on veut dire et écrire. A remplacer. Ici du texte et du blabla, ce que l'on veut dire et écrire. A remplacer.

\clearpage


%\chapter{Le titre du chapitre}

\section{Le titre de la section qui va bien}

\subsection{Titre de la sous section}

Ici du texte et du blabla, ce que l'on veut dire et écrire. A remplacer. Ici du texte et du blabla, ce que l'on veut dire et écrire. On peut faire une citation \cite{Motclef1}.
A remplacer. Ici du texte et du blabla, ce que l'on veut dire et écrire. A remplacer. Ici du texte et du blabla, ce que l'on veut dire et écrire. A remplacer. Ici du texte et du blabla, ce que l'on veut dire et écrire. A remplacer. Ici du texte et du blabla, ce que l'on veut dire et écrire. A remplacer.

Ici du texte et du blabla, ce que l'on veut dire et écrire. A remplacer. Ici du texte et du blabla, ce que l'on veut dire et écrire. A remplacer.
Ici du texte et du blabla, ce que l'on veut dire et écrire. A remplacer. Ici du texte et du blabla, ce que l'on veut dire et écrire. A remplacer. Ici du texte et du blabla, ce que l'on veut dire et écrire. A remplacer. Ici du texte et du blabla, ce que l'on veut dire et écrire. A remplacer.

%-- Note de bas de page sur les stades
\protect\footnote{Par exemple, on peut faire un pied de page :
\begin{itemize}
\item avec une liste à puces ;
\item avec une liste à puces ;
\item avec une liste à puces.
\end{itemize}
}
%-- Fin Note de bas de page sur les stades

Ici du texte et du blabla, ce que l'on veut dire et écrire. A remplacer. Ici du texte et du blabla, ce que l'on veut dire et écrire. A remplacer. Ici du texte et du blabla, ce que l'on veut dire et écrire. A remplacer. Ici du texte et du blabla, ce que l'on veut dire et écrire. A remplacer. Ici du texte et du blabla, ce que l'on veut dire et écrire. A remplacer. Ici du texte et du blabla, ce que l'on veut dire et écrire. A remplacer.

\begin{itemize}
\item avec une liste à puces ;
\item avec une liste à puces ;
\item avec une liste à puces.
\end{itemize}

Ici du texte et du blabla, ce que l'on veut dire et écrire. A remplacer. Ici du texte et du blabla, ce que l'on veut dire et écrire. A remplacer. Ici du texte et du blabla, ce que l'on veut dire et écrire. A remplacer. Ici du texte et du blabla, ce que l'on veut dire et écrire. A remplacer. Ici du texte et du blabla, ce que l'on veut dire et écrire. A remplacer. Ici du texte et du blabla, ce que l'on veut dire et écrire. A remplacer.

\subsubsection{Titre de la sous sous section}

Ici du texte et du blabla, ce que l'on veut dire et écrire. A remplacer. Ici du texte et du blabla, ce que l'on veut dire et écrire. A remplacer. Ici du texte et du blabla, ce que l'on veut dire et écrire. A remplacer. Ici du texte et du blabla, ce que l'on veut dire et écrire. A remplacer. Ici du texte et du blabla, ce que l'on veut dire et écrire. A remplacer. Ici du texte et du blabla, ce que l'on veut dire et écrire. A remplacer.

Ici du texte et du blabla, ce que l'on veut dire et écrire. A remplacer. Ici du texte et du blabla, ce que l'on veut dire et écrire. A remplacer. Ici du texte et du blabla, ce que l'on veut dire et écrire. A remplacer. Ici du texte et du blabla, ce que l'on veut dire et écrire. A remplacer. Ici du texte et du blabla, ce que l'on veut dire et écrire. A remplacer. Ici du texte et du blabla, ce que l'on veut dire et écrire. A remplacer.

\subsubsection{Titre de la sous sous section}

Ici du texte et du blabla, ce que l'on veut dire et écrire. A remplacer. Ici du texte et du blabla, ce que l'on veut dire et écrire. A remplacer. Ici du texte et du blabla, ce que l'on veut dire et écrire. A remplacer. Ici du texte et du blabla, ce que l'on veut dire et écrire. A remplacer. Ici du texte et du blabla, ce que l'on veut dire et écrire. A remplacer. Ici du texte et du blabla, ce que l'on veut dire et écrire. A remplacer.

Ici du texte et du blabla, ce que l'on veut dire et écrire. A remplacer. Ici du texte et du blabla, ce que l'on veut dire et écrire. A remplacer. Ici du texte et du blabla, ce que l'on veut dire et écrire. A remplacer. Ici du texte et du blabla, ce que l'on veut dire et écrire. A remplacer. Ici du texte et du blabla, ce que l'on veut dire et écrire. A remplacer. Ici du texte et du blabla, ce que l'on veut dire et écrire. A remplacer.

\subsection{Conclusion}

Ici du texte et du blabla, ce que l'on veut dire et écrire. A remplacer. Ici du texte et du blabla, ce que l'on veut dire et écrire. A remplacer. Ici du texte et du blabla, ce que l'on veut dire et écrire. A remplacer. Ici du texte et du blabla, ce que l'on veut dire et écrire. A remplacer. Ici du texte et du blabla, ce que l'on veut dire et écrire. A remplacer. Ici du texte et du blabla, ce que l'on veut dire et écrire. A remplacer.

Ici du texte et du blabla, ce que l'on veut dire et écrire. A remplacer. Ici du texte et du blabla, ce que l'on veut dire et écrire. A remplacer. Ici du texte et du blabla, ce que l'on veut dire et écrire. A remplacer. Ici du texte et du blabla, ce que l'on veut dire et écrire. A remplacer. Ici du texte et du blabla, ce que l'on veut dire et écrire. A remplacer. Ici du texte et du blabla, ce que l'on veut dire et écrire. A remplacer.

\subsection{Titre de la sous section}

Ici du texte et du blabla, ce que l'on veut dire et écrire. A remplacer. Ici du texte et du blabla, ce que l'on veut dire et écrire. On peut faire une citation \cite{Motclef1}.
A remplacer. Ici du texte et du blabla, ce que l'on veut dire et écrire. A remplacer. Ici du texte et du blabla, ce que l'on veut dire et écrire. A remplacer. Ici du texte et du blabla, ce que l'on veut dire et écrire. A remplacer. Ici du texte et du blabla, ce que l'on veut dire et écrire. A remplacer.

Ici du texte et du blabla, ce que l'on veut dire et écrire. A remplacer. Ici du texte et du blabla, ce que l'on veut dire et écrire. A remplacer.
Ici du texte et du blabla, ce que l'on veut dire et écrire. A remplacer. Ici du texte et du blabla, ce que l'on veut dire et écrire. A remplacer. Ici du texte et du blabla, ce que l'on veut dire et écrire. A remplacer. Ici du texte et du blabla, ce que l'on veut dire et écrire. A remplacer.

\subsection{Titre de la sous section}

Ici du texte et du blabla, ce que l'on veut dire et écrire. A remplacer. Ici du texte et du blabla, ce que l'on veut dire et écrire. On peut faire une citation \cite{Motclef1}.
A remplacer. Ici du texte et du blabla, ce que l'on veut dire et écrire. A remplacer. Ici du texte et du blabla, ce que l'on veut dire et écrire. A remplacer. Ici du texte et du blabla, ce que l'on veut dire et écrire. A remplacer. Ici du texte et du blabla, ce que l'on veut dire et écrire. A remplacer.

Ici du texte et du blabla, ce que l'on veut dire et écrire. A remplacer. Ici du texte et du blabla, ce que l'on veut dire et écrire. A remplacer.
Ici du texte et du blabla, ce que l'on veut dire et écrire. A remplacer. Ici du texte et du blabla, ce que l'on veut dire et écrire. A remplacer. Ici du texte et du blabla, ce que l'on veut dire et écrire. A remplacer. Ici du texte et du blabla, ce que l'on veut dire et écrire. A remplacer.

\clearpage

%
%\chapter{Le titre du chapitre}

\section{Le titre de la section qui va bien}

\subsection{Titre de la sous section}

Ici du texte et du blabla, ce que l'on veut dire et écrire. A remplacer. Ici du texte et du blabla, ce que l'on veut dire et écrire. On peut faire une citation \cite{Motclef1}.
A remplacer. Ici du texte et du blabla, ce que l'on veut dire et écrire. A remplacer. Ici du texte et du blabla, ce que l'on veut dire et écrire. A remplacer. Ici du texte et du blabla, ce que l'on veut dire et écrire. A remplacer. Ici du texte et du blabla, ce que l'on veut dire et écrire. A remplacer.

Ici du texte et du blabla, ce que l'on veut dire et écrire. A remplacer. Ici du texte et du blabla, ce que l'on veut dire et écrire. A remplacer.
Ici du texte et du blabla, ce que l'on veut dire et écrire. A remplacer. Ici du texte et du blabla, ce que l'on veut dire et écrire. A remplacer. Ici du texte et du blabla, ce que l'on veut dire et écrire. A remplacer. Ici du texte et du blabla, ce que l'on veut dire et écrire. A remplacer.

%-- Note de bas de page sur les stades
\protect\footnote{Par exemple, on peut faire un pied de page :
\begin{itemize}
\item avec une liste à puces ;
\item avec une liste à puces ;
\item avec une liste à puces.
\end{itemize}
}
%-- Fin Note de bas de page sur les stades

Ici du texte et du blabla, ce que l'on veut dire et écrire. A remplacer. Ici du texte et du blabla, ce que l'on veut dire et écrire. A remplacer. Ici du texte et du blabla, ce que l'on veut dire et écrire. A remplacer. Ici du texte et du blabla, ce que l'on veut dire et écrire. A remplacer. Ici du texte et du blabla, ce que l'on veut dire et écrire. A remplacer. Ici du texte et du blabla, ce que l'on veut dire et écrire. A remplacer.

\begin{itemize}
\item avec une liste à puces ;
\item avec une liste à puces ;
\item avec une liste à puces.
\end{itemize}

Ici du texte et du blabla, ce que l'on veut dire et écrire. A remplacer. Ici du texte et du blabla, ce que l'on veut dire et écrire. A remplacer. Ici du texte et du blabla, ce que l'on veut dire et écrire. A remplacer. Ici du texte et du blabla, ce que l'on veut dire et écrire. A remplacer. Ici du texte et du blabla, ce que l'on veut dire et écrire. A remplacer. Ici du texte et du blabla, ce que l'on veut dire et écrire. A remplacer.

\subsubsection{Titre de la sous sous section}

Ici du texte et du blabla, ce que l'on veut dire et écrire. A remplacer. Ici du texte et du blabla, ce que l'on veut dire et écrire. A remplacer. Ici du texte et du blabla, ce que l'on veut dire et écrire. A remplacer. Ici du texte et du blabla, ce que l'on veut dire et écrire. A remplacer. Ici du texte et du blabla, ce que l'on veut dire et écrire. A remplacer. Ici du texte et du blabla, ce que l'on veut dire et écrire. A remplacer.

Ici du texte et du blabla, ce que l'on veut dire et écrire. A remplacer. Ici du texte et du blabla, ce que l'on veut dire et écrire. A remplacer. Ici du texte et du blabla, ce que l'on veut dire et écrire. A remplacer. Ici du texte et du blabla, ce que l'on veut dire et écrire. A remplacer. Ici du texte et du blabla, ce que l'on veut dire et écrire. A remplacer. Ici du texte et du blabla, ce que l'on veut dire et écrire. A remplacer.

\subsubsection{Titre de la sous sous section}

Ici du texte et du blabla, ce que l'on veut dire et écrire. A remplacer. Ici du texte et du blabla, ce que l'on veut dire et écrire. A remplacer. Ici du texte et du blabla, ce que l'on veut dire et écrire. A remplacer. Ici du texte et du blabla, ce que l'on veut dire et écrire. A remplacer. Ici du texte et du blabla, ce que l'on veut dire et écrire. A remplacer. Ici du texte et du blabla, ce que l'on veut dire et écrire. A remplacer.

Ici du texte et du blabla, ce que l'on veut dire et écrire. A remplacer. Ici du texte et du blabla, ce que l'on veut dire et écrire. A remplacer. Ici du texte et du blabla, ce que l'on veut dire et écrire. A remplacer. Ici du texte et du blabla, ce que l'on veut dire et écrire. A remplacer. Ici du texte et du blabla, ce que l'on veut dire et écrire. A remplacer. Ici du texte et du blabla, ce que l'on veut dire et écrire. A remplacer.

\subsection{Conclusion}

Ici du texte et du blabla, ce que l'on veut dire et écrire. A remplacer. Ici du texte et du blabla, ce que l'on veut dire et écrire. A remplacer. Ici du texte et du blabla, ce que l'on veut dire et écrire. A remplacer. Ici du texte et du blabla, ce que l'on veut dire et écrire. A remplacer. Ici du texte et du blabla, ce que l'on veut dire et écrire. A remplacer. Ici du texte et du blabla, ce que l'on veut dire et écrire. A remplacer.

Ici du texte et du blabla, ce que l'on veut dire et écrire. A remplacer. Ici du texte et du blabla, ce que l'on veut dire et écrire. A remplacer. Ici du texte et du blabla, ce que l'on veut dire et écrire. A remplacer. Ici du texte et du blabla, ce que l'on veut dire et écrire. A remplacer. Ici du texte et du blabla, ce que l'on veut dire et écrire. A remplacer. Ici du texte et du blabla, ce que l'on veut dire et écrire. A remplacer.

\subsection{Titre de la sous section}

Ici du texte et du blabla, ce que l'on veut dire et écrire. A remplacer. Ici du texte et du blabla, ce que l'on veut dire et écrire. On peut faire une citation \cite{Motclef1}.
A remplacer. Ici du texte et du blabla, ce que l'on veut dire et écrire. A remplacer. Ici du texte et du blabla, ce que l'on veut dire et écrire. A remplacer. Ici du texte et du blabla, ce que l'on veut dire et écrire. A remplacer. Ici du texte et du blabla, ce que l'on veut dire et écrire. A remplacer.

Ici du texte et du blabla, ce que l'on veut dire et écrire. A remplacer. Ici du texte et du blabla, ce que l'on veut dire et écrire. A remplacer.
Ici du texte et du blabla, ce que l'on veut dire et écrire. A remplacer. Ici du texte et du blabla, ce que l'on veut dire et écrire. A remplacer. Ici du texte et du blabla, ce que l'on veut dire et écrire. A remplacer. Ici du texte et du blabla, ce que l'on veut dire et écrire. A remplacer.

\subsection{Titre de la sous section}

Ici du texte et du blabla, ce que l'on veut dire et écrire. A remplacer. Ici du texte et du blabla, ce que l'on veut dire et écrire. On peut faire une citation \cite{Motclef1}.
A remplacer. Ici du texte et du blabla, ce que l'on veut dire et écrire. A remplacer. Ici du texte et du blabla, ce que l'on veut dire et écrire. A remplacer. Ici du texte et du blabla, ce que l'on veut dire et écrire. A remplacer. Ici du texte et du blabla, ce que l'on veut dire et écrire. A remplacer.

Ici du texte et du blabla, ce que l'on veut dire et écrire. A remplacer. Ici du texte et du blabla, ce que l'on veut dire et écrire. A remplacer.
Ici du texte et du blabla, ce que l'on veut dire et écrire. A remplacer. Ici du texte et du blabla, ce que l'on veut dire et écrire. A remplacer. Ici du texte et du blabla, ce que l'on veut dire et écrire. A remplacer. Ici du texte et du blabla, ce que l'on veut dire et écrire. A remplacer.

\clearpage

%
%\chapter{Le titre du chapitre}

\section{Le titre de la section qui va bien}

\subsection{Titre de la sous section}

Ici du texte et du blabla, ce que l'on veut dire et écrire. A remplacer. Ici du texte et du blabla, ce que l'on veut dire et écrire. On peut faire une citation \cite{MotClef4}.
A remplacer. Ici du texte et du blabla, ce que l'on veut dire et écrire. A remplacer. Ici du texte et du blabla, ce que l'on veut dire et écrire. A remplacer. Ici du texte et du blabla, ce que l'on veut dire et écrire. A remplacer. Ici du texte et du blabla, ce que l'on veut dire et écrire. A remplacer.

Ici du texte et du blabla, ce que l'on veut dire et écrire. A remplacer. Ici du texte et du blabla, ce que l'on veut dire et écrire. A remplacer.
Ici du texte et du blabla, ce que l'on veut dire et écrire. A remplacer. Ici du texte et du blabla, ce que l'on veut dire et écrire. A remplacer. Ici du texte et du blabla, ce que l'on veut dire et écrire. A remplacer. Ici du texte et du blabla, ce que l'on veut dire et écrire. A remplacer.

%-- Note de bas de page sur les stades
\protect\footnote{Par exemple, on peut faire un pied de page :
\begin{itemize}
\item avec une liste à puces ;
\item avec une liste à puces ;
\item avec une liste à puces.
\end{itemize}
}
%-- Fin Note de bas de page sur les stades

Ici du texte et du blabla, ce que l'on veut dire et écrire. A remplacer. Ici du texte et du blabla, ce que l'on veut dire et écrire. A remplacer. Ici du texte et du blabla, ce que l'on veut dire et écrire. A remplacer. Ici du texte et du blabla, ce que l'on veut dire et écrire. A remplacer. Ici du texte et du blabla, ce que l'on veut dire et écrire. A remplacer. Ici du texte et du blabla, ce que l'on veut dire et écrire. A remplacer.

\begin{itemize}
\item avec une liste à puces ;
\item avec une liste à puces ;
\item avec une liste à puces.
\end{itemize}

Ici du texte et du blabla, ce que l'on veut dire et écrire. A remplacer. Ici du texte et du blabla, ce que l'on veut dire et écrire. A remplacer. Ici du texte et du blabla, ce que l'on veut dire et écrire. A remplacer. Ici du texte et du blabla, ce que l'on veut dire et écrire. A remplacer. Ici du texte et du blabla, ce que l'on veut dire et écrire. A remplacer. Ici du texte et du blabla, ce que l'on veut dire et écrire. A remplacer.

\subsubsection{Titre de la sous sous section}

Ici du texte et du blabla, ce que l'on veut dire et écrire. A remplacer. Ici du texte et du blabla, ce que l'on veut dire et écrire. A remplacer. Ici du texte et du blabla, ce que l'on veut dire et écrire. A remplacer. Ici du texte et du blabla, ce que l'on veut dire et écrire. A remplacer. Ici du texte et du blabla, ce que l'on veut dire et écrire. A remplacer. Ici du texte et du blabla, ce que l'on veut dire et écrire. A remplacer.

Ici du texte et du blabla, ce que l'on veut dire et écrire. A remplacer. Ici du texte et du blabla, ce que l'on veut dire et écrire. A remplacer. Ici du texte et du blabla, ce que l'on veut dire et écrire. A remplacer. Ici du texte et du blabla, ce que l'on veut dire et écrire. A remplacer. Ici du texte et du blabla, ce que l'on veut dire et écrire. A remplacer. Ici du texte et du blabla, ce que l'on veut dire et écrire. A remplacer.

\subsubsection{Titre de la sous sous section}

Ici du texte et du blabla, ce que l'on veut dire et écrire. A remplacer. Ici du texte et du blabla, ce que l'on veut dire et écrire. A remplacer. Ici du texte et du blabla, ce que l'on veut dire et écrire. A remplacer. Ici du texte et du blabla, ce que l'on veut dire et écrire. A remplacer. Ici du texte et du blabla, ce que l'on veut dire et écrire. A remplacer. Ici du texte et du blabla, ce que l'on veut dire et écrire. A remplacer.

Ici du texte et du blabla, ce que l'on veut dire et écrire. A remplacer. Ici du texte et du blabla, ce que l'on veut dire et écrire. A remplacer. Ici du texte et du blabla, ce que l'on veut dire et écrire. A remplacer. Ici du texte et du blabla, ce que l'on veut dire et écrire. A remplacer. Ici du texte et du blabla, ce que l'on veut dire et écrire. A remplacer. Ici du texte et du blabla, ce que l'on veut dire et écrire. A remplacer.

\subsection{Conclusion}

Ici du texte et du blabla, ce que l'on veut dire et écrire. A remplacer. Ici du texte et du blabla, ce que l'on veut dire et écrire. A remplacer. Ici du texte et du blabla, ce que l'on veut dire et écrire. A remplacer. Ici du texte et du blabla, ce que l'on veut dire et écrire. A remplacer. Ici du texte et du blabla, ce que l'on veut dire et écrire. A remplacer. Ici du texte et du blabla, ce que l'on veut dire et écrire. A remplacer.

Ici du texte et du blabla, ce que l'on veut dire et écrire. A remplacer. Ici du texte et du blabla, ce que l'on veut dire et écrire. A remplacer. Ici du texte et du blabla, ce que l'on veut dire et écrire. A remplacer. Ici du texte et du blabla, ce que l'on veut dire et écrire. A remplacer. Ici du texte et du blabla, ce que l'on veut dire et écrire. A remplacer. Ici du texte et du blabla, ce que l'on veut dire et écrire. A remplacer.

\subsection{Titre de la sous section}

Ici du texte et du blabla, ce que l'on veut dire et écrire. A remplacer. Ici du texte et du blabla, ce que l'on veut dire et écrire. On peut faire une citation \cite{MotClef4}.
A remplacer. Ici du texte et du blabla, ce que l'on veut dire et écrire. A remplacer. Ici du texte et du blabla, ce que l'on veut dire et écrire. A remplacer. Ici du texte et du blabla, ce que l'on veut dire et écrire. A remplacer. Ici du texte et du blabla, ce que l'on veut dire et écrire. A remplacer.

Ici du texte et du blabla, ce que l'on veut dire et écrire. A remplacer. Ici du texte et du blabla, ce que l'on veut dire et écrire. A remplacer.
Ici du texte et du blabla, ce que l'on veut dire et écrire. A remplacer. Ici du texte et du blabla, ce que l'on veut dire et écrire. A remplacer. Ici du texte et du blabla, ce que l'on veut dire et écrire. A remplacer. Ici du texte et du blabla, ce que l'on veut dire et écrire. A remplacer.

\subsection{Titre de la sous section}

Ici du texte et du blabla, ce que l'on veut dire et écrire. A remplacer. Ici du texte et du blabla, ce que l'on veut dire et écrire. On peut faire une citation \cite{MotClef4}.
A remplacer. Ici du texte et du blabla, ce que l'on veut dire et écrire. A remplacer. Ici du texte et du blabla, ce que l'on veut dire et écrire. A remplacer. Ici du texte et du blabla, ce que l'on veut dire et écrire. A remplacer. Ici du texte et du blabla, ce que l'on veut dire et écrire. A remplacer.

Ici du texte et du blabla, ce que l'on veut dire et écrire. A remplacer. Ici du texte et du blabla, ce que l'on veut dire et écrire. A remplacer.
Ici du texte et du blabla, ce que l'on veut dire et écrire. A remplacer. Ici du texte et du blabla, ce que l'on veut dire et écrire. A remplacer. Ici du texte et du blabla, ce que l'on veut dire et écrire. A remplacer. Ici du texte et du blabla, ce que l'on veut dire et écrire. A remplacer.

\clearpage


\chapter{Conclusion}

J'ai trouvé ce projet très intéressant. 
D'une part parce qu'il m'a fait manipuler la programmation réseau et 
également car il permet de travailler 
sur un code assez conséquent en langage C.
Il est probable que si nous avions eu un projet légèrement plus complet
que celui-ci je l'aurais structuré et découpé différemment.

Cependant, le fait d'avoir en même temps plusieurs projets (4) ainsi 
que des TP notés a été un frein et j'ai commencé le développement de 
ce projet que très tard par manque de temps.


% Pour finir l'interligne de 1,5
\end{onehalfspace}

%----------------------------------------
% Pour la bibliographie
%----------------------------------------
% Citer tous les ouvrages/références
\nocite{*}
% Trier par ordre d'apparition
%bibliographystyle{unsrt}
% Pour le style de la biblio
\bibliographystyle{plain.bst}
% Ecrire la biblio ici
\bibliography{biblio}

\printindex

\appendix


\end{document}
