\documentclass[12pt,a4paper,utf8x]{report}
\usepackage [frenchb]{babel}

% Pour pouvoir utiliser 
\usepackage{ucs}
\usepackage[utf8x]{inputenc}

\usepackage{url} % Pour avoir de belles url
\usepackage {geometry}

% Pour mettre du code source
\usepackage {listings}
% Pour pouvoir passer en paysage
\usepackage{lscape}

% Pour pouvoir faire plusieurs colonnes
\usepackage {multicol}
% POur crééer un index
\usepackage{makeidx}
\makeindex

% Pour les entetes de page
% \usepackage{fancyheadings}
%\pagestyle{fancy}
%\renewcommand{\sectionmark}[1]{\markboth{#1}{}} 
%\renewcommand{\subsectionmark}[1]{\markright{#1}} 

% Pour l'interligne de 1.5
\usepackage {setspace}
% Pour les marges de la page
\geometry{a4paper, top=2.5cm, bottom=3.5cm, left=1.5cm, right=1.5cm, marginparwidth=1.2cm}

\parskip=5pt %% distance entre § (paragraphe)
\sloppy %% respecter toujours la marge de droite 

% Pour les pénalités :
\interfootnotelinepenalty=150 %note de bas de page
\widowpenalty=150 %% veuves et orphelines
\clubpenalty=150 

%Pour la longueur de l'indentation des paragraphes
\setlength{\parindent}{15mm}



%%%% debut macro pour enlever le nom chapitre %%%%
\makeatletter
\def\@makechapterhead#1{%
  \vspace*{50\p@}%
  {\parindent \z@ \raggedright \normalfont
    \interlinepenalty\@M
    \ifnum \c@secnumdepth >\m@ne
        \Huge\bfseries \thechapter\quad
    \fi
    \Huge \bfseries #1\par\nobreak
    \vskip 40\p@
  }}

\def\@makeschapterhead#1{%
  \vspace*{50\p@}%
  {\parindent \z@ \raggedright
    \normalfont
    \interlinepenalty\@M
    \Huge \bfseries  #1\par\nobreak
    \vskip 40\p@
  }}
\makeatother
%%%% fin macro %%%%

%Couverture 

\title
{
	\normalsize{Rapport de projet\\
	Université de Strasbourg\\
	2012}\\
	\vspace{15mm}
	\Huge{Réseau et protocoles\\
Rapport de fin de projet}
}
\author{Pittoli Philippe
	\vspace{45mm}
}

\date{	
	\normalsize{
		20/05/2012
	\vspace{5mm}	
	}
}

\begin{document}

\maketitle

\tableofcontents
\clearpage

% Pour avoir un interligne de 1,5
\begin{onehalfspace}

\chapter{Introduction}

Voici mon rapport de fin de projet.
Il a été écrit avec \LaTeX ~et les sources sont disponibles sur le dépôt.

Le projet a consisté au développement d'une application de vente aux enchères en réseau.
Trois programmes distincts ont été développés, un programme de client enchérisseur, 
un programme destiné au vendeur et un serveur qui gère la vente.
Le langage utilisé a été le C.

Les sources sont également disponibles sur mon dépôt GIT \cite{GIT} public
\protect\footnote{
Github : https://github.com/KaneRoot/public
}.

%\chapter{}
\chapter{Le code}
\section{Organisation du code}
Mon code est séparé assez simplement, de manière classique le serveur et les clients ont chacun un fichier .c (le code), un .h (les prototypes) et un Makefile.
Ensuite nous avons le répertoire lib/ qui contient :
\begin{itemize}
\item messages.h, l'ensemble des messages affichés.
\item structures.h, les structures utilisées pour définir un message, les contextes
\protect\footnote{
	Un contexte est l'ensemble des variables globales réunies dans une structure pour 
	rendre le code plus propre et facile à lire.
}
, un produit ou les informations pour assurer l'identification
\protect\footnote{
	L'identification se fait via un nom d'utilisateur et un mot de passe chiffré.
} d'un client.
\item definitions.h, l'ensemble des constantes utilisées dans les trois programmes.
\item utilisateurs\_enregistrés : les enchérisseurs pour lesquels on accepte la connexion
\protect\footnote{
	Le fichier utilisateurs\_enregistrés est écrit de la manière suivante : "pseudo-de-connexion:mot-de-passe-chiffré".
	Voir le fichier "readme.md" pour connaître les utilisateurs déjà enregistrés.
}.
\end{itemize}


\section{Protocole}
Tous les messages transmis entre le serveur et les clients passent par l'envoi d'un "paquet"
\protect\footnote{
	Le "paquet" est une structure (message\_s) définie dans ./lib/definitions.h . 
	Le champ le plus important est "type" qui définit le type de message 
	(demande de connexion, envoi d'une offre sur un produit, etc.).
	C'est à partir de lui qu'on articule tout le reste du code.
}.
\subsection{Avant-propos}
Le client "bidder" prend deux paramètres (facultatifs) au lancement,
l'adresse du serveur et le port de connexion.
Si vous entrez une adresse IPv4 il se connectera en IPv4, sinon en v6.
Par défaut l'adresse est l'adresse locale, et le port 9000.
Si on ne rentre qu'un paramètre, ce sera l'adresse de connexion, 
   si on en rentre deux ce sera l'adresse suivie du port.

Le client "seller" prend les mêmes paramètres, mais ne se connecte qu'en IPv4.

Ceci nous donne :
\begin{itemize}
\item ./client\_bidder [IP-V4-OU-V6 [PORT]] ;
\item ./client\_seller [IP-V4 [PORT]] ;
\item ./server [PORT] ;
\end{itemize}


\subsection{La connexion}
\begin{enumerate}
\item On démarre le serveur.
	\begin{enumerate}
	\item Initialisation des sockets.
	\item Quelques variables sont initialisées.
	\end{enumerate}
\item On lance client\_seller .
	\begin{enumerate}
	\item Le vendeur se connecte.
	\item Le serveur lui attribue un numéro.
	\end{enumerate}
\item On lance une (ou plusieurs) fois client\_bidder.
	\begin{enumerate}
	\item Le client entre son nom et son mot de passe
\protect\footnote{
Le fichier lib/utilisateurs\_enregistres contient les informations sur les utilisateurs qui ont le droit de participer.
Et il est construit de la manière suivante: "nom de l'utilisateur:mot de passe chiffré".
}.
	\item Il envoie une requête de connexion au serveur.
	\item Le serveur lui répond (acceptation ou rejet).
		\begin{itemize}
		\item Si la connexion est acceptée, le client obtient un numéro de client.
		\end{itemize}
	\end{enumerate}
\end{enumerate}

\subsection{Une fois connecté}
Le fonctionnement du programme est relativement simple.

Tout s'articule autour de la commande "select".
Les deux types de clients ont leur entrée standard et le socket (udp ou tcp) 
de connexion au serveur qui sont écoutés par "select".
Le serveur écoute le socket udp et tcp de connexion ainsi que des sockets tcp de chaque vendeur.
Lors de la réception d'un message, on regarde le champ "type" du paquet via la commande "switch".
Le vendeur et l'enchérisseur ont quelques commandes définies.

\clearpage

Le vendeur peut :
\begin{itemize}
\item Ajouter des articles en vente.
\item Voir les enchères actuelles.
\item Démarrer les enchères.
\end{itemize}

Le client peut :
\begin{itemize}
\item Demander les articles en vente.
\item Faire une offre sur l'article courant.
\item Demander l'article actuellement en vente (l'article courant).
\end{itemize}

Si on écrit sur la fenêtre du serveur, on quitte le programme (et ses clients).

\subsection{Le déroulement des enchères}
Une fois que la connexion entre le serveur et les clients est établie, 
le vendeur envoie des objets à vendre à un certain prix.
Une fois fini, il fait démarrer les enchères.
À partir de là les clients sont informés que les enchères ont débuté.
Ils ont un temps défini pour envoyer leur offre (TEMPS\_ATTENTE dans definitions.h).
Si un client fait une offre valable (supérieure à la précédente offre et supérieure à l'offre de départ),
le compteur est remis au temps maximal.

Lorsque l'enchère est terminée pour un objet, tous les enchérisseurs sont informés.
Ils reçoivent également les informations à propos de l'article suivant.
Encore une fois, c'est au vendeur de débuter l'enchère sur cet article.

Une fois qu'il n'y a plus d'articles en vente, les enchères sont finies
et le vendeur reçoit le nom de la personne qui a remporté chaque article ainsi que son prix de vente et
également les objets non vendus.

\section{Démarche et difficultés techniques}

En premier lieu je me suis demandé comment le serveur allait gérer les clients, 
s'il allait falloir séparer le vendeur des enchérisseurs dans des structures de données différentes. 
Finalement j'ai préféré regrouper les deux dans une seule structure "client\_s". 
J'ai assez vite repris le code du premier TP avec l'utilisation du "select" qui est 
très pratique et permet de ne pas avoir à gérer de threads.

Toujours dans cette optique de ne pas avoir à gérer de threads, j'ai choisi d'utiliser un signal (SIGALRM)
pour la gestion du temps dans mon programme lorsque l'enchère a débuté.

J'ai pensé à faire de la compilation de bibliothèque partagée (déjà fait pour d'autres programmes) mais comme je n'ai
pas de fonctions similaires dans tous les programmes -mis à part "quitter" et "main"- cela ne m'a pas semblé nécessaire.

Il a fallu que j'apprenne à me servir de SVN, bien que ce n'était pas très compliqué et 
étant habitué de GIT cela ne m'a pas posé trop de soucis.

\subsection{Gestion de la pile IPv4 IPv6}

La double pile est gérée simplement. 
Sur le serveur j'ai deux sockets udp pour gérer les deux versions du protocole IP,
   et la commande "select" me permet de savoir quel est le socket sur lequel il y a de l'activité.

Concernant le stockage en mémoire des différentes adresses (v4 ou v6), 
		   ma structure client\_s possède les deux structures 
		   de données liées aux différentes adresses
\protect\footnote{ Les structures sockaddr\_in et sockaddr\_in6. }.
Pour savoir si je dois discuter avec le client en v4 ou en v6, 
je garde en mémoire la taille de son adresse
\protect\footnote{ Dans ma structure client\_s : l'attribut addrlen. }.

À savoir que le port d'écoute du serveur en UDP IPv6 est le numéro de port entré
en paramètre (ou celui par défaut) plus un.
Ceci n'est pas à prendre en compte lorsque l'on rentre le numéro de port en ligne de commande.
J'ai considéré ceci comme un détail d'implémentation.

J'ai également lu l'article de Stéphane Bortzmeyer \cite{BORTZMEYER} 
concernant la double couche IPv4 et v6, 
qui explique que la programmation en v6 est compatible avec la v4 nativement, 
mais que cette fonctionnalité est désactivée par défaut
\protect\footnote{ Lien vers l'article : http://www.bortzmeyer.org/bindv6only.html }.
Ceci aurait pu simplifier le programme.

\subsection{Tests}

J'ai pratiqué de nombreux tests en vérifiant les différents cas d'erreur possibles.
À chaque ajout de fonctionnalité j'ai recompilé puis testé 
l'intégralité des fonctions du programme.

%\protect\footnote{Git : gestionnaire de versionnement. } 

\chapter{Conclusion}

J'ai trouvé ce projet très intéressant. 
D'une part parce qu'il m'a fait manipuler la programmation réseau et 
également car il permet de travailler 
sur un code assez conséquent en langage C.
Il est probable que si nous avions eu un projet légèrement plus complet
que celui-ci je l'aurais structuré et découpé différemment.

Cependant, le fait d'avoir en même temps plusieurs projets (4) ainsi 
que des TP notés a été un frein et j'ai commencé le développement de 
ce projet que très tard par manque de temps.


% Pour finir l'interligne de 1,5
\end{onehalfspace}

%----------------------------------------
% Pour la bibliographie
%----------------------------------------
% Citer tous les ouvrages/références
\nocite{*}
% Trier par ordre d'apparition
%\bibliographystyle{unsrt}
% Pour le style de la biblio
\bibliographystyle{plain.bst}
% Ecrire la biblio ici
\bibliography{biblio}

\printindex

\appendix


\end{document}
