%\chapter{}
\chapter{Le code}
\section{Organisation du code}
Mon code est séparé assez simplement, de manière classique le serveur et les clients ont chacun un fichier .c (le code), un .h (les prototypes) et un Makefile.
Ensuite nous avons le répertoire lib/ qui contient :
\begin{itemize}
\item messages.h, l'ensemble des messages affichés.
\item structures.h, les structures utilisées pour définir un message, les contextes
\protect\footnote{
	Un contexte est l'ensemble des variables globales réunies dans une structure pour 
	rendre le code plus propre et facile à lire.
}
, un produit ou les informations pour assurer l'identification
\protect\footnote{
	L'identification se fait via un nom d'utilisateur et un mot de passe chiffré.
} d'un client.
\item definitions.h, l'ensemble des constantes utilisées dans les trois programmes.
\item utilisateurs\_enregistrés : les enchérisseurs pour lesquels on accepte la connexion
\protect\footnote{
	Le fichier utilisateurs\_enregistrés est écrit de la manière suivante : "pseudo-de-connexion:mot-de-passe-chiffré".
	Voir le fichier "readme.md" pour connaître les utilisateurs déjà enregistrés.
}.
\end{itemize}

