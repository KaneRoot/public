\section{Protocole}
Tous les messages transmis entre le serveur et les clients passent par l'envoi d'un "paquet"
\protect\footnote{
	Le "paquet" est une structure (message\_s) définie dans ./lib/definitions.h . 
	Le champ le plus important est "type" qui définit le type de message 
	(demande de connexion, envoi d'une offre sur un produit etc.).
	C'est à partir de lui qu'on articule tout le reste du code.
}.

\subsection{La connexion}
\begin{enumerate}
\item On démarre le serveur.
	\begin{enumerate}
	\item Initialisation des sockets.
	\item Quelques variables sont initialisées.
	\end{enumerate}
\item On lance client\_seller .
	\begin{enumerate}
	\item Le vendeur se connecte.
	\item Le serveur lui attribue un numéro.
	\end{enumerate}
\item On lance une (ou plusieurs) fois client\_bidder.
	\begin{enumerate}
	\item Le client entre son nom et son mot de passe
\protect\footnote{
Le fichier lib/utilisateurs\_enregistres contient les informations sur les utilisateurs qui ont le droit de participer.
Et il est construit de la manière suivante: "nom de l'utilisateur:mot de passe chiffré".
}.
	\item Il envoie une requête de connexion au serveur.
	\item Le serveur lui répond (acceptation ou rejet).
		\begin{itemize}
		\item Si la connexion est acceptée, le client obtient un numéro de client.
		\end{itemize}
	\end{enumerate}
\end{enumerate}

\subsection{Une fois connecté}
Le fonctionnement du programme est relativement simple.

Tout s'articule autour de la commande "select".
Les deux types de clients ont leur entrée standard et le socket (udp ou tcp) 
de connexion au serveur qui sont écoutés par "select".
Le serveur écoute le socket udp et tcp de connexion ainsi que des sockets tcp de chaque vendeur.
Lors de la réception d'un message, on regarde le champ "type" du paquet via la commande "switch".
Le vendeur et l'enchérisseur ont quelques commandes définies.

Le vendeur peut :
\begin{itemize}
\item Ajouter des articles en vente.
\item Voir les enchères actuelles.
\item Démarrer les enchères.
\end{itemize}

Le client peut :
\begin{itemize}
\item Demander les articles en vente.
\item Faire une offre sur l'article courant.
\item Demander l'article actuellement en vente (l'article courant).
\end{itemize}

\subsection{Le déroulement des enchères}
Une fois que la connexion entre le serveur et les clients est établie, 
le vendeur envoie des objets à vendre à un certain prix.
Une fois fini, il fait démarrer les enchères.
À partir de là les clients sont informés que les enchères ont débuté.
Ils ont un temps défini pour envoyer leur offre (TEMPS\_ATTENTE dans definitions.h).
Si un client fait une offre valable (supérieure à la précédente offre et supérieure à l'offre de départ),
le compteur est remis au temps maximal.

Lorsque l'enchère est terminée pour un objet, tous les enchérisseurs sont informés.
Ils reçoivent également les informations à propos de l'article suivant.
Encore une fois, c'est au vendeur de débuter l'enchère sur cet article.

Une fois qu'il n'y a plus d'articles en vente, les enchères sont finies
et le vendeur reçoit le nom de la personne qui a remporté chaque article ainsi que son prix de vente et
également les objets non vendus.
