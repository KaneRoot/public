\section{Démarche et difficultés techniques}

En premier lieu je me suis demandé comment le serveur allait gérer les clients, 
s'il allait falloir séparer le vendeur des enchérisseurs dans des structures de données différentes. 
Finalement j'ai préféré regrouper les deux dans une seule structure "client\_s". 
J'ai assez vite repris le code du premier TP avec l'utilisation du "select" qui est 
très pratique et permet de ne pas avoir à gérer de threads.

Toujours dans cette optique de ne pas avoir à gérer de threads, j'ai choisi d'utiliser un signal (SIGALRM)
pour la gestion du temps dans mon programme lorsque l'enchère a débuté.

J'ai pensé à faire de la compilation de bibliothèque partagée (déjà fait pour d'autres programmes) mais comme je n'ai
pas de fonctions similaires dans tous les programmes -mis à part "quitter" et "main"- cela ne m'a pas semblé nécessaire.

Il a fallu que j'apprenne à me servir de SVN, bien que ce n'était pas très compliqué et 
étant habitué de GIT cela ne m'a pas posé trop de soucis.

\subsection{Gestion de la pile IPv4 IPv6}

Pas de gestion de la double pile.
Mon programme est cependant fonctionnel et répond au cahier des charges.

J'ai également lu l'article de Stéphane Bortzmeyer \cite{BORTZMEYER} 
concernant la double couche IPv4 et v6, 
qui explique que la programmation en v6 est compatible avec la v4 nativement, 
mais que cette fonctionnalité est désactivée par défaut
\protect\footnote{ Lien vers l'article : http://www.bortzmeyer.org/bindv6only.html }.

\subsection{Tests}

J'ai pratiqué de nombreux tests en vérifiant les différents cas d'erreur possibles.
À chaque ajout de fonctionnalité j'ai recompilé puis testé 
l'intégralité des fonctions du programme.

%\protect\footnote{Git : gestionnaire de versionnement. } 
