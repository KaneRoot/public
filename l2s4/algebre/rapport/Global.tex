\documentclass[12pt,a4paper,utf8x]{report}
\usepackage [frenchb]{babel}

% Pour pouvoir utiliser 
\usepackage{ucs}
\usepackage[utf8x]{inputenc}

\usepackage{url} % Pour avoir de belles url
\usepackage {geometry}

% Pour mettre du code source
\usepackage {listings}
% Pour pouvoir passer en paysage
\usepackage{lscape}

% Pour pouvoir faire plusieurs colonnes
\usepackage {multicol}
% POur crééer un index
\usepackage{makeidx}
\makeindex

% Pour les entetes de page
% \usepackage{fancyheadings}
%\pagestyle{fancy}
%\renewcommand{\sectionmark}[1]{\markboth{#1}{}} 
%\renewcommand{\subsectionmark}[1]{\markright{#1}} 

% Pour l'interligne de 1.5
\usepackage {setspace}
% Pour les marges de la page
\geometry{a4paper, top=2.5cm, bottom=3.5cm, left=1.5cm, right=1.5cm, marginparwidth=1.2cm}

\parskip=5pt %% distance entre § (paragraphe)
\sloppy %% respecter toujours la marge de droite 

% Pour les pénalités :
\interfootnotelinepenalty=150 %note de bas de page
\widowpenalty=150 %% veuves et orphelines
\clubpenalty=150 

%Pour la longueur de l'indentation des paragraphes
\setlength{\parindent}{15mm}



%%%% debut macro pour enlever le nom chapitre %%%%
\makeatletter
\def\@makechapterhead#1{%
  \vspace*{50\p@}%
  {\parindent \z@ \raggedright \normalfont
    \interlinepenalty\@M
    \ifnum \c@secnumdepth >\m@ne
        \Huge\bfseries \thechapter\quad
    \fi
    \Huge \bfseries #1\par\nobreak
    \vskip 40\p@
  }}

\def\@makeschapterhead#1{%
  \vspace*{50\p@}%
  {\parindent \z@ \raggedright
    \normalfont
    \interlinepenalty\@M
    \Huge \bfseries  #1\par\nobreak
    \vskip 40\p@
  }}
\makeatother
%%%% fin macro %%%%

%Couverture 

\title
{
	\normalsize{Université de Strasbourg\\
	2011-2012}\\
	\vspace{15mm}
	\Huge{Rapport de projet}
}
\author{Philippe Pittoli\\
	\vspace{45mm}
}

%\date{	
%	\normalsize{Lieu du stage\\
%	Adresse du stage\\
%	Ville du stage\\ 
%	\vspace{5mm}	
%	Directeur de recherche : M. DUPONT \\
%	Rapporteur universitaire : Mme DUPUIS
%	}
%}

\begin{document}

\maketitle

%Remerciements

Je tiens à remercier :
et on met la liste des personnes que l'on remercie. Toto, tutu, titi. et on met la liste des personnes que l'on remercie. Toto, tutu, titi.et on met la liste des personnes que l'on remercie. Toto, tutu, titi.et on met la liste des personnes que l'on remercie. Toto, tutu, titi.


Et on met la liste des personnes que l'on remercie. Toto, tutu, titi.et on met la liste des personnes que l'on remercie. Toto, tutu, titi.et on met la liste des personnes que l'on remercie. Toto, tutu, titi.et on met la liste des personnes que l'on remercie. Toto, tutu, titi.et on met la liste des personnes que l'on remercie. Toto, tutu, titi.

%\clearpage

\tableofcontents
\clearpage

% Pour avoir un interligne de 1,5
\begin{onehalfspace}

\chapter{Introduction}

Voici mon rapport de fin de projet.
Il a été écrit avec \LaTeX ~et les sources sont disponibles sur le dépôt.

Le projet a consisté au développement d'une application de vente aux enchères en réseau.
Trois programmes distincts ont été développés, un programme de client enchérisseur, 
un programme destiné au vendeur et un serveur qui gère la vente.
Le langage utilisé a été le C.

Les sources sont également disponibles sur mon dépôt GIT \cite{GIT} public
\protect\footnote{
Github : https://github.com/KaneRoot/public
}.


\chapter{Les fichiers du projet}

Les noms des fichiers sont assez explicites pour ne pas avoir à expliquer plus en détail leur contenu. 
Cependant si vous souhaitez en savoir plus, les fichiers ".h" sont suffisamment commentés pour servir de manuel.

\section{Les sources du programme}

Voici une liste simple des fichiers du projet.


\begin{itemize}
	\item base.c
	\item base.h
\end{itemize}
Permet la création des différentes structures, les polynômes, les matrices et les matrices de polynômes.
Permet également la destruction propre de ces mêmes structures et leur affichage.

\begin{itemize}
	\item calculs base.c
	\item calculs base.h
\end{itemize}
Permet de faire des additions, multiplications de lignes sur une matrice (éventuellement de polynômes) ainsi que des additions et multiplications de polynômes.

\begin{itemize}
	\item determinant.c
	\item determinant.h
\end{itemize}
Calcul du déterminant d'une matrice (det nxn).

\begin{itemize}
	\item comatrice.c
	\item comatrice.h
\end{itemize}
Calcul de la comatrice (utilise le déterminant).

\begin{itemize}
	\item pivot-gauss.c
	\item pivot-gauss.h
\end{itemize}
Effectue le pivot de Gauss sur une matrice en répercutant chaque addition de lignes, inversion de lignes etc. sur une seconde matrice passée en parallèle.
Une fonction présente également est la recherche du pivot suivant dans une matrice à un emplacement donné.

\begin{itemize}
	\item inversion.c
	\item inversion.h
\end{itemize}
Calcule l'inversion d'une matrice par pivot de Gauss et par comatrice.

\begin{itemize}
	\item valeurs-propres.c
	\item valeurs-propres.h
\end{itemize}
Calcul des valeurs propres.
Probablement la fonction la plus complexe. Fonctionnement : recherche une ligne ou une colonne avec deux zéros, s'il y a alors on calcule les valeurs propres sinon on renvoit une matrice nulle.

\begin{itemize}
	\item tests.c
\end{itemize}
L'ensemble des tests possibles sont présents dans ce fichier.
Certains tests sont d'ailleurs présents sans option dans le menu pour y accéder (car utiles uniquement pour des phases de développement un peu précis).

\begin{itemize}
	\item Makefile
\end{itemize}
Produit un unique exécutable via la commande "make".

\section{Les sources du rapport}


\begin{itemize}
	\item biblio.bib
	\item Conclusion.tex
	\item Fichiers.tex
	\item Global.pdf
	\item Global.tex
	\item Intro.tex
	\item Tests.tex
\end{itemize}

\clearpage


\chapter{Les Tests}

Pour rappel, tous les tests sont présents dans un seul fichier : tests.c .
Tout d'abord, je décris comment faire fonctionner le programme, puis je donne quelques détails concernant mon programme.

\section{Comment faire marcher le programme}

L'ensemble des fonctions est testé via un seul fichier exécutable. Il n'y a qu'à faire make ; ./tests .
Il n'y a qu'à lire ce qui est affiché à l'écran.
Un menu a été mis en place pour naviguer entre les différents tests possibles.

\section{Concernant les sources du programme de tests}
\subsection{Les procédures de confort}

Voici donc une liste des fonctions que j'utilise à des fins de confort : 
\begin{itemize}
	\item choix-nbl-nbc
	\item creation-matrice
	\item print-menu
	\item menu
	\item afficher poly vide ou pas
\end{itemize}

Ces fonctions sont simples et me permettent soit de factoriser mon code soit d'unifier l'affichage.

\subsection{Les procédures de test à proprement parler}
Les fonctions plus importantes sont affichées dans le menu lorsque l'on démarre le programme.
D'autres sont présentes mais ne m'ont servi qu'à faire des tests unitaires de développement.

Chaque test demande de créer une ou deux matrices. Le reste est affiché petit à petit.

\clearpage


%\chapter{Le titre du chapitre}

\section{Le titre de la section qui va bien}

\subsection{Titre de la sous section}

Ici du texte et du blabla, ce que l'on veut dire et écrire. A remplacer. Ici du texte et du blabla, ce que l'on veut dire et écrire. On peut faire une citation \cite{Motclef1}.
A remplacer. Ici du texte et du blabla, ce que l'on veut dire et écrire. A remplacer. Ici du texte et du blabla, ce que l'on veut dire et écrire. A remplacer. Ici du texte et du blabla, ce que l'on veut dire et écrire. A remplacer. Ici du texte et du blabla, ce que l'on veut dire et écrire. A remplacer.

Ici du texte et du blabla, ce que l'on veut dire et écrire. A remplacer. Ici du texte et du blabla, ce que l'on veut dire et écrire. A remplacer.
Ici du texte et du blabla, ce que l'on veut dire et écrire. A remplacer. Ici du texte et du blabla, ce que l'on veut dire et écrire. A remplacer. Ici du texte et du blabla, ce que l'on veut dire et écrire. A remplacer. Ici du texte et du blabla, ce que l'on veut dire et écrire. A remplacer.

%-- Note de bas de page sur les stades
\protect\footnote{Par exemple, on peut faire un pied de page :
\begin{itemize}
\item avec une liste à puces ;
\item avec une liste à puces ;
\item avec une liste à puces.
\end{itemize}
}
%-- Fin Note de bas de page sur les stades

Ici du texte et du blabla, ce que l'on veut dire et écrire. A remplacer. Ici du texte et du blabla, ce que l'on veut dire et écrire. A remplacer. Ici du texte et du blabla, ce que l'on veut dire et écrire. A remplacer. Ici du texte et du blabla, ce que l'on veut dire et écrire. A remplacer. Ici du texte et du blabla, ce que l'on veut dire et écrire. A remplacer. Ici du texte et du blabla, ce que l'on veut dire et écrire. A remplacer.

\begin{itemize}
\item avec une liste à puces ;
\item avec une liste à puces ;
\item avec une liste à puces.
\end{itemize}

Ici du texte et du blabla, ce que l'on veut dire et écrire. A remplacer. Ici du texte et du blabla, ce que l'on veut dire et écrire. A remplacer. Ici du texte et du blabla, ce que l'on veut dire et écrire. A remplacer. Ici du texte et du blabla, ce que l'on veut dire et écrire. A remplacer. Ici du texte et du blabla, ce que l'on veut dire et écrire. A remplacer. Ici du texte et du blabla, ce que l'on veut dire et écrire. A remplacer.

\subsubsection{Titre de la sous sous section}

Ici du texte et du blabla, ce que l'on veut dire et écrire. A remplacer. Ici du texte et du blabla, ce que l'on veut dire et écrire. A remplacer. Ici du texte et du blabla, ce que l'on veut dire et écrire. A remplacer. Ici du texte et du blabla, ce que l'on veut dire et écrire. A remplacer. Ici du texte et du blabla, ce que l'on veut dire et écrire. A remplacer. Ici du texte et du blabla, ce que l'on veut dire et écrire. A remplacer.

Ici du texte et du blabla, ce que l'on veut dire et écrire. A remplacer. Ici du texte et du blabla, ce que l'on veut dire et écrire. A remplacer. Ici du texte et du blabla, ce que l'on veut dire et écrire. A remplacer. Ici du texte et du blabla, ce que l'on veut dire et écrire. A remplacer. Ici du texte et du blabla, ce que l'on veut dire et écrire. A remplacer. Ici du texte et du blabla, ce que l'on veut dire et écrire. A remplacer.

\subsubsection{Titre de la sous sous section}

Ici du texte et du blabla, ce que l'on veut dire et écrire. A remplacer. Ici du texte et du blabla, ce que l'on veut dire et écrire. A remplacer. Ici du texte et du blabla, ce que l'on veut dire et écrire. A remplacer. Ici du texte et du blabla, ce que l'on veut dire et écrire. A remplacer. Ici du texte et du blabla, ce que l'on veut dire et écrire. A remplacer. Ici du texte et du blabla, ce que l'on veut dire et écrire. A remplacer.

Ici du texte et du blabla, ce que l'on veut dire et écrire. A remplacer. Ici du texte et du blabla, ce que l'on veut dire et écrire. A remplacer. Ici du texte et du blabla, ce que l'on veut dire et écrire. A remplacer. Ici du texte et du blabla, ce que l'on veut dire et écrire. A remplacer. Ici du texte et du blabla, ce que l'on veut dire et écrire. A remplacer. Ici du texte et du blabla, ce que l'on veut dire et écrire. A remplacer.

\subsection{Conclusion}

Ici du texte et du blabla, ce que l'on veut dire et écrire. A remplacer. Ici du texte et du blabla, ce que l'on veut dire et écrire. A remplacer. Ici du texte et du blabla, ce que l'on veut dire et écrire. A remplacer. Ici du texte et du blabla, ce que l'on veut dire et écrire. A remplacer. Ici du texte et du blabla, ce que l'on veut dire et écrire. A remplacer. Ici du texte et du blabla, ce que l'on veut dire et écrire. A remplacer.

Ici du texte et du blabla, ce que l'on veut dire et écrire. A remplacer. Ici du texte et du blabla, ce que l'on veut dire et écrire. A remplacer. Ici du texte et du blabla, ce que l'on veut dire et écrire. A remplacer. Ici du texte et du blabla, ce que l'on veut dire et écrire. A remplacer. Ici du texte et du blabla, ce que l'on veut dire et écrire. A remplacer. Ici du texte et du blabla, ce que l'on veut dire et écrire. A remplacer.

\subsection{Titre de la sous section}

Ici du texte et du blabla, ce que l'on veut dire et écrire. A remplacer. Ici du texte et du blabla, ce que l'on veut dire et écrire. On peut faire une citation \cite{Motclef1}.
A remplacer. Ici du texte et du blabla, ce que l'on veut dire et écrire. A remplacer. Ici du texte et du blabla, ce que l'on veut dire et écrire. A remplacer. Ici du texte et du blabla, ce que l'on veut dire et écrire. A remplacer. Ici du texte et du blabla, ce que l'on veut dire et écrire. A remplacer.

Ici du texte et du blabla, ce que l'on veut dire et écrire. A remplacer. Ici du texte et du blabla, ce que l'on veut dire et écrire. A remplacer.
Ici du texte et du blabla, ce que l'on veut dire et écrire. A remplacer. Ici du texte et du blabla, ce que l'on veut dire et écrire. A remplacer. Ici du texte et du blabla, ce que l'on veut dire et écrire. A remplacer. Ici du texte et du blabla, ce que l'on veut dire et écrire. A remplacer.

\subsection{Titre de la sous section}

Ici du texte et du blabla, ce que l'on veut dire et écrire. A remplacer. Ici du texte et du blabla, ce que l'on veut dire et écrire. On peut faire une citation \cite{Motclef1}.
A remplacer. Ici du texte et du blabla, ce que l'on veut dire et écrire. A remplacer. Ici du texte et du blabla, ce que l'on veut dire et écrire. A remplacer. Ici du texte et du blabla, ce que l'on veut dire et écrire. A remplacer. Ici du texte et du blabla, ce que l'on veut dire et écrire. A remplacer.

Ici du texte et du blabla, ce que l'on veut dire et écrire. A remplacer. Ici du texte et du blabla, ce que l'on veut dire et écrire. A remplacer.
Ici du texte et du blabla, ce que l'on veut dire et écrire. A remplacer. Ici du texte et du blabla, ce que l'on veut dire et écrire. A remplacer. Ici du texte et du blabla, ce que l'on veut dire et écrire. A remplacer. Ici du texte et du blabla, ce que l'on veut dire et écrire. A remplacer.

\clearpage


%\chapter{Le titre du chapitre}

\section{Le titre de la section qui va bien}

\subsection{Titre de la sous section}

Ici du texte et du blabla, ce que l'on veut dire et écrire. A remplacer. Ici du texte et du blabla, ce que l'on veut dire et écrire. On peut faire une citation \cite{MotClef4}.
A remplacer. Ici du texte et du blabla, ce que l'on veut dire et écrire. A remplacer. Ici du texte et du blabla, ce que l'on veut dire et écrire. A remplacer. Ici du texte et du blabla, ce que l'on veut dire et écrire. A remplacer. Ici du texte et du blabla, ce que l'on veut dire et écrire. A remplacer.

Ici du texte et du blabla, ce que l'on veut dire et écrire. A remplacer. Ici du texte et du blabla, ce que l'on veut dire et écrire. A remplacer.
Ici du texte et du blabla, ce que l'on veut dire et écrire. A remplacer. Ici du texte et du blabla, ce que l'on veut dire et écrire. A remplacer. Ici du texte et du blabla, ce que l'on veut dire et écrire. A remplacer. Ici du texte et du blabla, ce que l'on veut dire et écrire. A remplacer.

%-- Note de bas de page sur les stades
\protect\footnote{Par exemple, on peut faire un pied de page :
\begin{itemize}
\item avec une liste à puces ;
\item avec une liste à puces ;
\item avec une liste à puces.
\end{itemize}
}
%-- Fin Note de bas de page sur les stades

Ici du texte et du blabla, ce que l'on veut dire et écrire. A remplacer. Ici du texte et du blabla, ce que l'on veut dire et écrire. A remplacer. Ici du texte et du blabla, ce que l'on veut dire et écrire. A remplacer. Ici du texte et du blabla, ce que l'on veut dire et écrire. A remplacer. Ici du texte et du blabla, ce que l'on veut dire et écrire. A remplacer. Ici du texte et du blabla, ce que l'on veut dire et écrire. A remplacer.

\begin{itemize}
\item avec une liste à puces ;
\item avec une liste à puces ;
\item avec une liste à puces.
\end{itemize}

Ici du texte et du blabla, ce que l'on veut dire et écrire. A remplacer. Ici du texte et du blabla, ce que l'on veut dire et écrire. A remplacer. Ici du texte et du blabla, ce que l'on veut dire et écrire. A remplacer. Ici du texte et du blabla, ce que l'on veut dire et écrire. A remplacer. Ici du texte et du blabla, ce que l'on veut dire et écrire. A remplacer. Ici du texte et du blabla, ce que l'on veut dire et écrire. A remplacer.

\subsubsection{Titre de la sous sous section}

Ici du texte et du blabla, ce que l'on veut dire et écrire. A remplacer. Ici du texte et du blabla, ce que l'on veut dire et écrire. A remplacer. Ici du texte et du blabla, ce que l'on veut dire et écrire. A remplacer. Ici du texte et du blabla, ce que l'on veut dire et écrire. A remplacer. Ici du texte et du blabla, ce que l'on veut dire et écrire. A remplacer. Ici du texte et du blabla, ce que l'on veut dire et écrire. A remplacer.

Ici du texte et du blabla, ce que l'on veut dire et écrire. A remplacer. Ici du texte et du blabla, ce que l'on veut dire et écrire. A remplacer. Ici du texte et du blabla, ce que l'on veut dire et écrire. A remplacer. Ici du texte et du blabla, ce que l'on veut dire et écrire. A remplacer. Ici du texte et du blabla, ce que l'on veut dire et écrire. A remplacer. Ici du texte et du blabla, ce que l'on veut dire et écrire. A remplacer.

\subsubsection{Titre de la sous sous section}

Ici du texte et du blabla, ce que l'on veut dire et écrire. A remplacer. Ici du texte et du blabla, ce que l'on veut dire et écrire. A remplacer. Ici du texte et du blabla, ce que l'on veut dire et écrire. A remplacer. Ici du texte et du blabla, ce que l'on veut dire et écrire. A remplacer. Ici du texte et du blabla, ce que l'on veut dire et écrire. A remplacer. Ici du texte et du blabla, ce que l'on veut dire et écrire. A remplacer.

Ici du texte et du blabla, ce que l'on veut dire et écrire. A remplacer. Ici du texte et du blabla, ce que l'on veut dire et écrire. A remplacer. Ici du texte et du blabla, ce que l'on veut dire et écrire. A remplacer. Ici du texte et du blabla, ce que l'on veut dire et écrire. A remplacer. Ici du texte et du blabla, ce que l'on veut dire et écrire. A remplacer. Ici du texte et du blabla, ce que l'on veut dire et écrire. A remplacer.

\subsection{Conclusion}

Ici du texte et du blabla, ce que l'on veut dire et écrire. A remplacer. Ici du texte et du blabla, ce que l'on veut dire et écrire. A remplacer. Ici du texte et du blabla, ce que l'on veut dire et écrire. A remplacer. Ici du texte et du blabla, ce que l'on veut dire et écrire. A remplacer. Ici du texte et du blabla, ce que l'on veut dire et écrire. A remplacer. Ici du texte et du blabla, ce que l'on veut dire et écrire. A remplacer.

Ici du texte et du blabla, ce que l'on veut dire et écrire. A remplacer. Ici du texte et du blabla, ce que l'on veut dire et écrire. A remplacer. Ici du texte et du blabla, ce que l'on veut dire et écrire. A remplacer. Ici du texte et du blabla, ce que l'on veut dire et écrire. A remplacer. Ici du texte et du blabla, ce que l'on veut dire et écrire. A remplacer. Ici du texte et du blabla, ce que l'on veut dire et écrire. A remplacer.

\subsection{Titre de la sous section}

Ici du texte et du blabla, ce que l'on veut dire et écrire. A remplacer. Ici du texte et du blabla, ce que l'on veut dire et écrire. On peut faire une citation \cite{MotClef4}.
A remplacer. Ici du texte et du blabla, ce que l'on veut dire et écrire. A remplacer. Ici du texte et du blabla, ce que l'on veut dire et écrire. A remplacer. Ici du texte et du blabla, ce que l'on veut dire et écrire. A remplacer. Ici du texte et du blabla, ce que l'on veut dire et écrire. A remplacer.

Ici du texte et du blabla, ce que l'on veut dire et écrire. A remplacer. Ici du texte et du blabla, ce que l'on veut dire et écrire. A remplacer.
Ici du texte et du blabla, ce que l'on veut dire et écrire. A remplacer. Ici du texte et du blabla, ce que l'on veut dire et écrire. A remplacer. Ici du texte et du blabla, ce que l'on veut dire et écrire. A remplacer. Ici du texte et du blabla, ce que l'on veut dire et écrire. A remplacer.

\subsection{Titre de la sous section}

Ici du texte et du blabla, ce que l'on veut dire et écrire. A remplacer. Ici du texte et du blabla, ce que l'on veut dire et écrire. On peut faire une citation \cite{MotClef4}.
A remplacer. Ici du texte et du blabla, ce que l'on veut dire et écrire. A remplacer. Ici du texte et du blabla, ce que l'on veut dire et écrire. A remplacer. Ici du texte et du blabla, ce que l'on veut dire et écrire. A remplacer. Ici du texte et du blabla, ce que l'on veut dire et écrire. A remplacer.

Ici du texte et du blabla, ce que l'on veut dire et écrire. A remplacer. Ici du texte et du blabla, ce que l'on veut dire et écrire. A remplacer.
Ici du texte et du blabla, ce que l'on veut dire et écrire. A remplacer. Ici du texte et du blabla, ce que l'on veut dire et écrire. A remplacer. Ici du texte et du blabla, ce que l'on veut dire et écrire. A remplacer. Ici du texte et du blabla, ce que l'on veut dire et écrire. A remplacer.

\clearpage


\chapter{Conclusion}

J'ai trouvé ce projet très intéressant. 
D'une part parce qu'il m'a fait manipuler la programmation réseau et 
également car il permet de travailler 
sur un code assez conséquent en langage C.
Il est probable que si nous avions eu un projet légèrement plus complet
que celui-ci je l'aurais structuré et découpé différemment.

Cependant, le fait d'avoir en même temps plusieurs projets (4) ainsi 
que des TP notés a été un frein et j'ai commencé le développement de 
ce projet que très tard par manque de temps.


% Pour finir l'interligne de 1,5
\end{onehalfspace}

%----------------------------------------
% Pour la bibliographie
%----------------------------------------
% Citer tous les ouvrages/références
\nocite{*}
% Trier par ordre d'apparition
%bibliographystyle{unsrt}
% Pour le style de la biblio
\bibliographystyle{plain.bst}
% Ecrire la biblio ici
\bibliography{biblio}

\printindex

\appendix


\end{document}
