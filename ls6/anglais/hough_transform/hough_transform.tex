%%%%%%%%%%%%%%%%%%%%%%%%%%%%%%%%%%%%%%%%%
% Journal Article
% LaTeX Template
% Version 1.0 (25/8/12)
%
% This template has been downloaded from:
% http://www.LaTeXTemplates.com
%
% Original author:
% Frits Wenneker (http://www.howtotex.com)
%
% License:
% CC BY-NC-SA 3.0 (http://creativecommons.org/licenses/by-nc-sa/3.0/)
%
%%%%%%%%%%%%%%%%%%%%%%%%%%%%%%%%%%%%%%%%%

%----------------------------------------------------------------------------------------
%	PACKAGES AND OTHER DOCUMENT CONFIGURATIONS
%----------------------------------------------------------------------------------------

\documentclass[twoside]{article}

% rajout personnel
\usepackage{graphicx}

\usepackage{lipsum} % Package to generate dummy text throughout this template

\usepackage[sc]{mathpazo} % Use the Palatino font
\usepackage[T1]{fontenc} % Use 8-bit encoding that has 256 glyphs
\linespread{1.05} % Line spacing - Palatino needs more space between lines
\usepackage{microtype} % Slightly tweak font spacing for aesthetics

\usepackage[hmarginratio=1:1,top=32mm,columnsep=20pt]{geometry} % Document margins
\usepackage{multicol} % Used for the two-column layout of the document
\usepackage[colorlinks=true, urlcolor=cyan, pdftitle={Hough Transform}]{hyperref}

\usepackage[hang, small,labelfont=bf,up,textfont=it,up]{caption} % Custom captions under/above floats in tables or figures
\usepackage{booktabs} % Horizontal rules in tables
\usepackage{float} % Required for tables and figures in the multi-column environment - they need to be placed in specific locations with the [H] (e.g. \begin{table}[H])

\usepackage{lettrine} % The lettrine is the first enlarged letter at the beginning of the text
\usepackage{paralist} % Used for the compactitem environment which makes bullet points with less space between them

\usepackage{abstract} % Allows abstract customization
\renewcommand{\abstractnamefont}{\normalfont\bfseries} % Set the "Abstract" text to bold
\renewcommand{\abstracttextfont}{\normalfont\small\itshape} % Set the abstract itself to small italic text

\usepackage{titlesec} % Allows customization of titles
\titleformat{\section}[block]{\large\scshape\centering{\Roman{section}.}}{}{1em}{} % Change the look of the section titles 

\usepackage{fancyhdr} % Headers and footers
\pagestyle{fancy} % All pages have headers and footers
\fancyhead{} % Blank out the default header
\fancyfoot{} % Blank out the default footer
\fancyhead[C]{Hough Transform$\bullet$ December 2012} % Custom header text
\fancyfoot[RO,LE]{\thepage} % Custom footer text

%----------------------------------------------------------------------------------------
%	TITLE SECTION
%----------------------------------------------------------------------------------------

\title{\vspace{-15mm}\fontsize{24pt}{10pt}\selectfont\textbf{Hough Transform}} % Article title

\author{
\large
\textsc{Philippe Pittoli}\\[2mm] % Your name
\normalsize University of Strasbourg \\ % Your institution
\normalsize \href{mailto:philippe.pittoli@etu.unistra.fr}{contact-me} % Your email address
\vspace{-5mm}
}
\date{}

%----------------------------------------------------------------------------------------

\begin{document}

\maketitle % Insert title

\thispagestyle{fancy} % All pages have headers and footers

%----------------------------------------------------------------------------------------
%	ARTICLE CONTENTS
%----------------------------------------------------------------------------------------

\section{What is the purpose of the Hough transform ?}
The Hough transform is a technique used in many domains that can find imperfect instances of objects within a certain class of shapes.
It was originally used to identify lines on images, now it has been extended to identify positions of arbitrary shapes.

\subsection{On which type of images is it used ?}
It's used on images with edges that can be well detected with as less noise as possible to be efficient (the wikipedia and the other articles don't seem to agree on that point, they probably talk about different conditions or usage of the algorithm).
It could be used on an image with a point cloud.

\section{How is the Hough transform used to detect linear alignments ?}
The Hough transform aims to address the shape detection problem by succeeding in performing groupings of edge points into object candidates thanks to a kind of voting procedure.
The votes fall in different bins (computer array) according to detect the existence of a line as $y = mx + b$.
The value of a bin increases whenever the Hough transform algorithm determines that there is enough evidence of an edge at a pixel or their neighbors.

\subsection{The parametrization of lines.}
The main idea is to consider the characteristics of the straight line not as image points $(x1, y1), (x2, y2)$, etc.,
but to use instead a different pair of parameters, $r$ and $\theta$, for the lines in the Hough transform, called the Polar Coordinates.
The parameter $r$ represents the distance between the line and the origin, $\theta$ is the angle of the vector from the origin to this closest point.
The equation of the line can be written as $r = x \cos \theta+y\sin \theta$.

\section{How are the best candidates for lines selected ?}
The best candidates that could be chosen are points that form perfect lines without any noises, so that the algorithm will be the most efficient as possible.
Each vote will fall in an unique bin.
Otherwise, if it is too noisy, a de-noising stage will have to be done before starting the procedure.

\subsection{What technique does one use to limit the computing load ?}
The Hough Transform is only efficient if, during the voting procedure, a high number of votes fall in the right bin, so that the bin can be easily detected among the background noise. 
This means that the bin must not be too small. If it is, some votes will fall in the neighboring bins, and that will reduce the visibility of the main bin.
One way to limit the computing load could be to arrange to have as less noise as possible.

\section{Can other shapes be detected by a type of Hough transform ?}
There are few more shapes that can be detected thanks to this algorithm, such as planes, cylinders, and other analytical and non-analytical shapes.
Even a circle can be transformed into a set of three parameters, and the Hough transform can be adapted to accept three parameters and to detect this shape.

\end{document}
